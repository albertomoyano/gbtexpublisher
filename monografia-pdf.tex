\documentclass[12pt,twoside]{article}

% Diseño de la página: define las dimensiones de la página y los márgenes.
\usepackage[paperwidth=210mm,
paperheight=297mm,
top=15mm,
includehead,
headheight=3\baselineskip,
bottom=20mm,
left=20mm,
marginparsep=6mm,
marginparwidth=44mm,
textwidth=13cm,
footskip=20pt,
footnotesep=13.5pt plus 0.1pt minus 0.1pt]{geometry}


% Configuración de idiomas y opciones de Babel: permite utilizar múltiples idiomas en el documento.
\usepackage[french,portuguese,italian,english,german,spanish,es-ucroman,es-noshorthands]{babel}

% Estilo bibliográfico APA
\usepackage[style=ieee]{biblatex}

%% Estilo bibliográfico IEEE
%\usepackage[style=ieee]{biblatex}
%
%% Estilo bibliográfico Vancouver
%\usepackage[style=ieee]{biblatex}
%
%% Estilo bibliográfico Oxford (nota a pie)
%\usepackage[style=ieee]{biblatex}
%
%% Estilo bibliográfico Verona (autor/año)
%\usepackage[style=ieee]{biblatex}

\defcounter{biburlnumpenalty}{3000}
\defcounter{biburlucpenalty}{6000}
\defcounter{biburllcpenalty}{9000}

% Soporte para fuentes OpenType y TrueType: permite la utilización de fuentes personalizadas.
\usepackage{fontspec}

% Mejoras tipográficas: mejora la calidad tipográfica, como la justificación y el espaciado.
\usepackage{microtype}

% Configuración de microtipografía: personaliza la microtipografía con ajustes específicos.
\newfontfeature{Microtypography}{protrusion=default;expansion=default}
\directlua{fonts.protrusions.setups.default.factor=.5}

% Redefinición del tamaño de fuente normal: ajusta el tamaño y el interlineado del texto.
\renewcommand{\normalsize}{\fontsize{11pt}{13.5pt}\selectfont}
\topskip=13.5pt

% Tipografía principal Libertinus: define la fuente principal del documento.
\setmainfont{Libertinus Serif}
[Numbers={OldStyle,Proportional},Ligatures=TeX,Scale=1.18]

% Soporte para símbolos matemáticos: utiliza la tipografía Libertinus para las matemáticas.
\usepackage{unicode-math}
\setmathfont{Libertinus Math}

% Configuración para el idioma chino con tipografía predeterminada: soporte para escritura en chino.
\usepackage{luatexja-fontspec}
\setmainjfont{FandolSong}

% Tipografía sans-serif y monoespaciada: define las fuentes para texto sans-serif y monoespaciado.
\setsansfont[Scale=MatchLowercase,
Ligatures=TeX,
Extension=.otf,
UprightFont=*-Regular,
ItalicFont=*-Italic,
BoldFont=*-SemiBold,
BoldItalicFont=*-SemiBoldItalic]{IBMPlexSansCondensed}

\setmonofont[Scale=0.91,
Extension=.otf,
UprightFont=*-Regular,
ItalicFont = IBMPlexMono-Italic.otf,
BoldFont = IBMPlexMono-Bold.otf,
BoldItalicFont = IBMPlexMono-BoldItalic.otf]{IBMPlexMono.otf}

% El paquete 'froufrou' proporciona decoraciones para mejorar la presentación del texto.
\usepackage{froufrou}

% El paquete 'booktabs' facilita la creación de tablas de calidad profesional, proporcionando comandos para mejorar el diseño de las tablas.
\usepackage{booktabs}

% El paquete 'rotating' permite rotar objetos como figuras, tablas y texto en cualquier ángulo, útil para presentar contenido en orientaciones diferentes.
\usepackage{rotating}

% El paquete 'graphicx' es fundamental para la inclusión de imágenes en un documento LaTeX, permitiendo redimensionar, rotar y recortar imágenes.
\usepackage{graphicx}

% El paquete 'svg' permite la inclusión directa de gráficos SVG en el documento, mejorando la calidad visual y la escalabilidad de los gráficos vectoriales.
\usepackage{svg}

% El paquete 'pdfpages' facilita la inclusión de páginas completas de archivos PDF dentro de un documento LaTeX, ideal para incorporar documentos externos.
\usepackage[final]{pdfpages}

% El paquete 'caption' proporciona herramientas para personalizar las leyendas de figuras y tablas, incluyendo el formato, la fuente, y la posición de la etiqueta.
\usepackage[labelfont=bf,font=small,labelsep=period,format=plain]{caption}

% El paquete 'ragged2e' ofrece opciones avanzadas de justificación del texto, permitiendo alinearlo a la izquierda, derecha o centrado, sin justificarlo completamente.
\usepackage{ragged2e}

% El paquete 'xcolor' permite utilizar colores en el texto, fondos y gráficos dentro del documento, con una amplia gama de opciones de personalización.
\usepackage{xcolor}

% Definición de colores personalizados: establece colores que pueden ser usados en el documento.
\definecolor{color1}{RGB}{0,0,0}
\definecolor{color2}{RGB}{144,12,63}

% El paquete 'mdframed', con el método 'tikz', permite crear cuadros alrededor de secciones de texto o gráficos, personalizables con bordes, fondos y más.
\usepackage[framemethod=tikz]{mdframed}

% El paquete 'bchart' permite crear gráficos de barras simples y personalizables dentro de un documento LaTeX, ideal para visualizaciones básicas de datos.
\usepackage{bchart}

% El paquete 'tcolorbox', con la opción 'most', proporciona cajas de texto coloreadas y decoradas para resaltar secciones importantes, con muchas opciones de personalización.
\usepackage[most]{tcolorbox}

% El paquete 'marginnote' permite agregar notas en el margen del documento, útil para comentarios o anotaciones que no interfieren con el texto principal.
\usepackage{marginnote}

% El paquete 'ifthen' ofrece herramientas para realizar evaluaciones condicionales dentro del documento, permitiendo ejecutar comandos basados en ciertas condiciones.
\usepackage{ifthen}

\usepackage{lastpage}

% Configuración de secciones y títulos: define el estilo y el espaciado de los títulos de secciones.
\usepackage[sf,bf,compact,topmarks,calcwidth,pagestyles,clearempty,newlinetospace]{titlesec}

% Diseño de cabecera
\renewpagestyle{plain}[]{
\sethead[][][]{}{}{}
\setfoot{}{}{}}

\newpagestyle{myps}[]{
    \sethead[][][]{
                  \begin{minipage}[b]{.2\textwidth}
                  \raggedright
                  \includegraphics[height=20mm]{\logo}
                  \end{minipage}
                  }{}{
                     \begin{minipage}[b]{.8\textwidth+\marginparsep+\marginparwidth}
                     \raggedleft
                     \textsc{\MakeLowercase\autor}\\
                     \textsc{\MakeLowercase\materia}\\
                    \textsc{\fecha}
                    \end{minipage}
                    }
    \setfoot[pág.~\textbf{\usepage} de \pageref{LastPage}]
    []
    []
    {}
    {}
    {pág.~\textbf{\usepage} de \pageref{LastPage}}
}
\pagestyle{myps}

% Diseño de la sección principal: formato del título de las secciones.
\titleformat{\section}[hang]
{\sf\bfseries\large\raggedright\color{color2}}
{\thesection}{.5em}{}[]

% Espaciado y margen de las secciones principales.
\titlespacing{\section}
{\parindent}{18pt plus 0.5pt minus 0.5pt}{6.75pt}

% Diseño de subsecciones: formato del título de las subsecciones.
\titleformat{\subsection}[hang]
{\sf\large\raggedright}
{\thesubsection}{.5em}{}[]

% Espaciado y margen de las subsecciones.
\titlespacing{\subsection}
{\parindent}{18pt plus 0.5pt minus 0.5pt}{6.75pt}

% Diseño de subsubsecciones: formato del título de las subsubsecciones.
\titleformat{\subsubsection}[hang]
{\rm\bfseries\normalsize\raggedright}
{\thesubsubsection}{.5em}{}[]

% Espaciado y margen de las subsubsecciones.
\titlespacing{\subsubsection}
{\parindent}{18pt plus 0.5pt minus 0.5pt}{6.75pt}

% Diseño de párrafos: formato de los párrafos.
\titleformat{\paragraph}[runin]
  {\bfseries}
  {}{0em}{}
  [\mbox{ --- }]

% Espaciado y margen de los párrafos.
\titlespacing{\paragraph}
  {0pt}% antes de la raya
  {6.75pt plus 0.5pt minus 0.5pt}% antes del párrafo
  {0pt}% después de la raya

% Diseño de subpárrafos: formato de los subpárrafos.
\titleformat{\subparagraph}[hang]
{\sf\bfseries\large\centering}
{\thesubparagraph}{.5em}{}[]

% Espaciado y margen de los subpárrafos.
\titlespacing{\subparagraph}
{\parindent}{18pt plus 0.5pt minus 0.5pt}{6.75pt}

% Configuraciones directas para la maquetación del documento.
\raggedbottom % Evita el estiramiento vertical de las páginas.
\clubpenalty=10000 % Evita líneas huérfanas al final de una página.
\widowpenalty=10000 % Evita líneas viudas al comienzo de una página.

% Permite tener un control más preciso sobre el corte de la url
\RequirePackage{url}
\Urlmuskip = 0mu plus 1mu
\def\UrlBreaks{\do\a\do\b\do\c\do\d\do\e\do\f\do\g\do\h\do\i\do\j\do\k\do\l\do\m\do\n\do\o\do\p\do\q\do\r\do\s\do\t\do\u\do\v\do\w\do\x\do\y\do\z\do\A\do\B\do\C\do\D\do\E\do\F\do\G\do\H\do\I\do\J\do\K\do\L\do\M\do\N\do\O\do\P\do\Q\do\R\do\S\do\T\do\U\do\V\do\W\do\X\do\Y\do\Z\do0\do1\do2\do3\do4\do5\do6\do7\do8\do9\do=\do.\do:\do\%\do?\do_\do-\do+\do/\do\#\do~}
\def\UrlFont{\rm}

% Paquete necesario para \mathbb
\usepackage{amssymb}

% Paquete necesario para los entornos matemáticos
\usepackage{amsthm}

% Definición de entornos matemáticos personalizados
\newtheorem{theorem}{Teorema}
\newtheorem{lemma}[theorem]{Lema}
\newtheorem{proposition}[theorem]{Proposición}
\newtheorem{corollary}[theorem]{Corolario}
\newtheorem{definition}{Definición}
\newtheorem{example}{Ejemplo}
\newtheorem{axiom}{Axioma}
\newtheorem{conjecture}{Conjetura}
\theoremstyle{remark}% ---> necesita amsthm
\newtheorem{remark}{Observación}

% Hipervínculos y referencias: permite enlaces y referencias dentro del documento.
\usepackage[allcolors=blue,colorlinks,unicode]{hyperref}


