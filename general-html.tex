\documentclass{book}
\usepackage[
HomeHTMLFilename=index,% Filename of the homepage.
HTMLFilename={node-},% Filename prefix of other pages.
mathjax,%Use MathJax to display math.
xindy,
latexmk]{lwarp}
\usepackage{lipsum}

\usepackage[spanish]{babel}
\usepackage{csquotes}
\usepackage{graphicx}
\usepackage[backend=biber,backref=true,indexing=cite,citestyle=authoryear,style=authoryear]{biblatex}
%\usepackage[xindy,toc,numberedsection=nolabel]{glossaries}
\usepackage[xindy,acronym,sanitizesort,toc=true,nonumberlist]{glossaries}
\makenoidxglossaries

%hyperref siempre debe quedar último
\usepackage{hyperref}
% termina preámbulo predefinido

%comienza preámbulo dinámico desde la base de datos
\boolfalse{FileSectionNames}% If false, numbers the files.
\renewcommand{\linkhomename}{Inicio}
\renewcommand{\linkpreviousname}{Anterior}
\renewcommand{\linknextname}{Siguiente}
\renewcommand{\sidetocname}{Sumario}
\setcounter{tocdepth}{2}% Include subsections in the \TOC.
\setcounter{secnumdepth}{2}% Number down to subsections.
\setcounter{FileDepth}{1}% Split \HTML\ files at sections
\setcounter{FootnoteDepth}{1}
\booltrue{CombineHigherDepths}% Combine parts/chapters/sections
\setcounter{SideTOCDepth}{1}% Include subsections in the side\TOC
\HTMLTitle{Después de parir}% Overrides \title for the web page.
\HTMLAuthor{Alberto Moyano}% Sets the HTML meta author tag.
\HTMLLanguage{es-ES}% Sets the HTML meta language.
\HTMLDescription{Cada página de salida HTML debe tener su propia meta descripción HTML, que usualmente aparece en los resultados de búsqueda en la web.}% Generalmente, está limitada a alrededor de 150 caracteres de longitud y no debe incluir el carácter de comillas dobles ASCII.}% Sets the HTML meta description.
\HTMLFirstPageTop{\includegraphics[width=5cm]{./media/cover.png}}
%\HTMLPageTop{Después de parir}
\HTMLFirstPageBottom{Información de contacto y legales (\url{https://www.edicionesimagomundi.com/})}
\HTMLPageBottom{\LinkPrevious | \LinkNext}
\HTMLKeywords{clave1,clave2,clave3}
\HTMLAddMeta{book-title}{Después de parir}
\HTMLAddMeta{subtitle}{Que dolor}
\HTMLAddMeta{trans-title}{After birthing}
\HTMLAddMeta{edition}{Primera}
\HTMLAddMeta{publisher-name}{Ediciones Imago Mundi}
\HTMLAddMeta{publisher-loc}{Buenos Aires}
\HTMLAddMeta{isbn}{000-000-0000-000-0}
\HTMLAddMeta{pub-date}{2024/05/17}
\HTMLAddMeta{volume}{1}
\HTMLAddMeta{series-title}{Colección Bitácora Argentina}
\HTMLAddMeta{copyright-statement}{© 2023 Example Publisher}
\HTMLAddMeta{license-type}{Tipo de licencia: CC BY}
\HTMLAddMeta{license-text}{Creative Commons Attribution 4.0 International License}
\HTMLAddMeta{funding-source}{Fuente de financiamiento: National Science Foundation}
\HTMLAddMeta{abstract}{Cada página de salida HTML debe tener su propia meta descripción HTML, que usualmente aparece en los resultados de búsqueda en la web. Generalmente, está limitada a alrededor de 150 caracteres de longitud y no debe incluir el carácter de comillas dobles ASCII.}
\CSSFilename{imago_custom.css}

% fin del preambulo
